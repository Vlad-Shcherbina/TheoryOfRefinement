\documentclass{article}
\usepackage{amssymb}
\usepackage{amsmath}
\usepackage{amsthm}
\usepackage{color}

\begin{document}

\newcommand{\refines}{\sqsubseteq}
\newcommand{\refinedby}{\sqsupseteq}
\newcommand{\eps}{\varepsilon}

% to hide proofs
%\renewenvironment{proof}{\tiny Proof.}{\qed}

% pretty arrows
%\newcommand{\plotkin}[3]{#1\stackrel{#2}{\rightarrow}#3}
%\newcommand{\milner}[3]{#1\stackrel{#2}{\leadsto}#3}

% ugly arrows
\newcommand{\plotkin}[3]{\left< #1, #2 \right> \rightarrow #3}
\newcommand{\milner}[3]{#1-#2\rightarrow #3}


\newtheorem{axiom}{Axiom}
\newtheorem{law}{Law}
\newtheorem{rul}{Rule}  % because 'rule' is already occupied
\newtheorem{theorem}{Theorem}
\newtheorem{lemma}{Lemma}
\newtheorem{definition}{Definition}

Laws and rules are mere shortcuts for some statements.
Only theorems are supposed to be proven (and they typically state that
some rule implies some law or vice versa).


\begin{axiom} [$\refines$ is partial order]
\end{axiom}

\begin{axiom} [Monotonicity of $;$]
\end{axiom}


\begin{definition} [Hoare triple]
$\{p\}q\{r\}$ means $p;q \refines r$
\end{definition}

\begin{definition} [Plotkin reduction]
$\plotkin{p}{q}{r}$ means $p; q \refinedby r$
\end{definition}

\begin{definition} [Milner transition]
$\milner{p}{q}{r}$ means $p \refinedby q; r$
\end{definition}

\begin{definition} [Test generation semantics]
$p[q]r$ means $p \refines q;r$
\end{definition}


\subsection*{Associativity and sequence rules}

\begin{law}[Assoc. of ;]
$p;(q;r) = (p;q);r$
\end{law}

\begin{rul}[Sequence rule for Hoare]
$\{p\}q\{s\}$ and $\{s\}q'\{r\}$ imply $\{p\}q;q'\{r\}$
\end{rul}

\begin{rul}[Sequence rule for Plotkin]
$\plotkin{p}{q}{s}$ and $\plotkin{s}{q'}{r}$ imply $\plotkin{p}{(q; q')}{r}$
\end{rul}

\begin{rul}[Sequence rule for Milner]
$\milner{p}{q}{s}$ and $\milner{s}{q'}{r}$ imply $\milner{p}{(q; q')}{r}$
\end{rul}

\begin{rul}[Sequence rule for testgen]
$p [q] s$ and $s [q] r$ imply $p [q; q'] r$
\end{rul}

\begin{theorem}
Associativity law implies all sequence rules.
\end{theorem}

{\color{green}Verified.}

\begin{proof}
Hoare.\\
Let $\{p\}q\{s\}$ and $\{s\}q'\{r\}$.\\
According to the definition it means $p; q \refines s$ and $s; q' \refines r$\\
We are to prove that $p; (q; q') \refines r$\\
$p; (q; q') =^{assoc.} (p; q); q' \refines^{mon.} s; q' \refines r$\\
\\
Plotkin.\\
Let $\plotkin{p}{q}{s}$ and $\plotkin{s}{q'}{r}$.\\
According to the definition it means $s \refines p; q$ and $r \refines s; q'$\\
We are to prove that $r \refines p; (q; q')$\\
$r \refines s; q' \refines^{mon.} (p; q); q' =^{assoc.} p; (q; q')$\\
\\
Milner.\\
Let $\milner{p}{q}{s}$ and $\milner{s}{q'}{r}$.\\
According to the definition it means $q; s \refines p; q$ and $q'; r \refines s$\\
We are to prove that $(q; q'); r \refines p$\\
$(q; q'); r =^{assoc.} q; (q'; r) \refines^{mon.} q; s \refines p$\\
\\
Testgen.\\
Let $p [q] s$ and $s [q'] r$\\
According to the definition it means $p \refines q; s$ and $s \refines q'; r$\\
We are to prove that $p \refines (q; q'); r$\\
$p \refines q; s \refines^{mon.} q; (q'; r) =^{assoc.} (q; q'); r$
\end{proof}

\begin{theorem}
Sequence rules for Hoare + Plotkin or Milner + testgen imply associativity law.
\end{theorem}

\begin{proof}
Hoare + Plotkin.\\
Let $\{p\}q\{s\}$ and $\{s\}q'\{r\}$ imply $\{p\}q;q'\{r\}$\\
and\\
$\plotkin{p}{q}{s}$ and $\plotkin{s}{q'}{r}$ imply $\plotkin{p}{(q;q')}{r}$\\
According to the definition it means\\
$p; q \refines s$ and $s; q' \refines r$ imply $p; (q; q') \refines r$\\
and\\
$s \refines p; q$ and $r \refines s; q'$ imply $r \refines p; (q; q')$\\
As $\refines$ is partial order then $p; q = s$ and $s; q' = r$\\
Also $p; (q; q') = r$.\\
$p; (q; q') = r = s; q' = (p; q); q'$.\\
\\
Milner + testgen.\\
Let $\milner{p}{q}{s}$ and $\milner{s}{q'}{r}$ imply $\milner{p}{(q;q')}{r}$\\
and\\
$p [q] s$ and $s [q'] r$ imply $p [q; q'] r$\\
According to the definition it means\\
$q; s \refines p$ and $q'; r \refines s$ imply $(q; q'); r \refines p$\\
and\\
$p \refines q; s$ and $s \refines q'; r$ imply $p \refines (q; q'); r$\\
As $\refines$ is partial order then $p = q; s$ and $s = q'; r$\\
Also $p = (q; q'); r$.\\
$(q; q'); r = q; s = q; (q'; r)$
\end{proof}

\subsection*{Skip}

\begin{law}[Skip]
$p; \eps = p = \eps; p$
\end{law}

\begin{rul}[Nullity for Hoare]
$\{p\} \eps \{p\}$
\end{rul}

\begin{rul}[Nullity for Plotkin]
$\plotkin{p}{\eps}{p}$
\end{rul}

\begin{rul}[Nullity for Milner]
$\milner{p}{\eps}{p}$
\end{rul}

\begin{rul}[Nullity for testgen]
$p [\eps] p$
\end{rul}

\begin{theorem}
Skip law implies all nullity rules.
\end{theorem}

\begin{proof}
As $\refines$ is reflexive\\
$p \refines p$\\
$p; \eps \refines p$\\
$\{p\} \eps \{p\}$\\
\\
$p \refines p$\\
$p; \refines p; \eps$\\
$\plotkin{p}{\eps}{p}$\\
\\
$p \refines p$\\
$\eps; p \refines p$\\
$\milner{p}{\eps}{p}$\\
\\
$p \refines p$\\
$p \refines \eps; p$\\
$p [\eps] p$
\end{proof}

\begin{theorem}
Conjunction of all nullity rules implies skip law.
\end{theorem}

\begin{proof}
$\{p\} \eps \{p\}$ implies $p; \eps \refines p$\\
$\plotkin{p}{\eps}{p}$ implies $p; \refines p; \eps$\\
As $\refines$ is partial order then $p = p; \eps$\\
$\milner{p}{\eps}{p}$ implies $\eps; p; \refines p$\\
$p [\eps] p$ implies $p \refines \eps; p$\\
As $\refines$ is partial order then $p = \eps; p$\\
So we have $p = p; \eps = \eps; p$
\end{proof}

\subsection*{Distribution and nondeterminism}

\begin{definition}[$\sqcup$]
$p \sqcup q$ is the smallest $r$ such that $p \refines r$ and $q \refines r$.
\end{definition}

\begin{law}[Left distributivity]
$p; (q \sqcup q') = p; q \sqcup p; q'$
\end{law}

\begin{law}[Right distributivity]
$(q \sqcup q'); r = q; r \sqcup q'; r$
\end{law}

\begin{lemma}
Non-determinism is commutative, increasing and idempotent.
\end{lemma}

{\color{green}Verified.}

\begin{proof}
Commutativity is obvious (directly by definition).\\
\\
Let's prove $p \refines p \sqcup q$ for any $q$.\\
$p \sqcup q = r$ where $r$ is the smallest that $p \refines r$ and $q \refines r$.\\
$p \refines r = p \sqcup q$.\\
\\
Let's prove $p \sqcup p = p$.\\
$p \sqcup p = r$ where $r$ is the smallest that $p \refines r$.\\
As $\refines$ is reflexive then $p \refines p$ and $p$ is the smallest.
$p = r = p \sqcup p$.
\end{proof}

\begin{rul}[Nondeterminism for Hoare]
$\{p\}q\{r\}$ and $\{p\}q'\{r\}$ imply $\{p\} q \sqcup q' \{r\}$
\end{rul}

\begin{rul}[Nondeterminism for Plotkin]
$\plotkin{p}{q}{r}$ or $\plotkin{p}{q'}{r}$ imply $\plotkin{p}{(q \sqcup q')}{r}$
\end{rul}

\begin{rul}[Nondeterminism for Milner]
$\milner{p}{q}{r}$ and $\milner{p}{q'}{r}$ imply $\milner{p}{(q \sqcup q')}{r}$
\end{rul}

\begin{rul}[Nondeterminism for testgen]
$p [q] r$ or $p [q'] r$ imply $p [q \sqcup q'] r$
\end{rul}

\begin{theorem}
Left distributivity implies nondeterminism for Hoare and Plotkin.
\end{theorem}

{\color{green}Verified 1/2.}

\begin{proof}
Hoare.\\
Let $p; q \refines r$ and $p; q' \refines r$.\\
We have to proove $p; (q \sqcup q') \refines r$.\\
$p; (q \sqcup q') = p; q \sqcup p; q' \refines^{mon.} r \sqcup r =^{idemp.} r$.\\
\\
Plotkin.\\
Let $r \refines p; q$ (1) or $r \refines p; q'$ (2).\\
We have to proove $r \refines p; (q \sqcup q')$.\\
1. $r \refines^{increas.} r \sqcup p; q' \refines^{mon.} p; q \sqcup p; q' =^{law} p; (q \sqcup q')$\\
2. $r \refines^{increas.} r \sqcup p; q =^{commut.} p; q \sqcup r \refines^{mon.} p; q \sqcup p; q' =^{law} p; (q \sqcup q')$\\
\end{proof}

\begin{theorem}
Right distributivity implies nondeterminism for Milner and testgen.
\end{theorem}

\begin{proof}
Milner.\\
Let $q; r \refines p$ and $q'; r \refines p$.\\
We have to proove $(q \sqcup q'); r \refines p$.\\
$(q \sqcup q'); r = q; r \sqcup q'; r \refines^{mon} p \sqcup p =^{idemp.} = p$.\\
\\
Testgen.\\
Let $p \refines q; r$ (1) or $p \refines q'; r$ (2).\\
We have to proove $p \refines (q \sqcup q'); r$.\\
1. $p \refines p \sqcup q'; r \refines q; r \sqcup q'; r = (q \sqcup q'); r$\\
2. $p \refines p \sqcup q; r = q; r \sqcup p \refines q; r \sqcup q'; r = (q \sqcup q'); r$\\
\end{proof}

\begin{theorem}
Nondeterminism rules for Hoare and Plotkin imply left distributivity.
\end{theorem}

\begin{proof}
Let $p; q \refines r$ and $p; q' \refines r$ imply $p; (q \sqcup q') \refines r$\\
and $r \refines p; q$ (1) or $r \refines p; q'$ (2) imply $r \refines p; (q \sqcup q')$.\\
$r = p; (q \sqcup q')$\\
1. $r \refines r \sqcup p; q' \refines p; q \sqcup p; q' \refines r \sqcup r = r$\\
Then $r = p; q \sqcup p; q'$.\\
2. $r \refines r \sqcup p; q = p; q \sqcup r \refines p; q \sqcup p; q' \refines r \sqcup r = r$\\
Then $r = p; q \sqcup p; q'$.
\end{proof}

\begin{theorem}
Nondeterminism rules for Milner and testgen imply right distributivity.
\end{theorem}

\begin{proof}
Let $q; r \refines p$ and $q'; r \refines p$ imply $(q \sqcup q'); r \refines p$\\
and $p \refines q; r$ (1) or $p \refines q'; r$ (2) imply $p \refines (q \sqcup q'); r$.\\
$p = (q \sqcup q'); r$\\
1. $p \refines p \sqcup q'; r \refines q; r \sqcup q'; r \refines p \sqcup p = p$\\
Then $p = q; r \sqcup q'; r$.\\
2. $p \refines p \sqcup q; r = q; r \sqcup p \refines q; r \sqcup q'; r \refines p \sqcup p = p$\\
Then $p = q; r \sqcup q'; r$.\\
\end{proof}

\subsection*{Locality and frame rules}

\begin{law}[Left locality]
$(s | p); q \refines s | (p; q)$
\end{law}

\begin{law}[Right locality]
$p; (q | s) \refines (p; q) | s$
\end{law}

\begin{rul}[Hoare frame rule]
$\{p\} q \{r\}$ implies $\{p | s\} q \{r | s\}$
\end{rul}

\begin{rul}[Milner frame rule]
$\milner{p}{q}{r}$ implies $\milner{p | s}{q}{r | s}$
\end{rul}

\begin{theorem}
Left locality implies Hoare frame rule.
\end{theorem}

\begin{proof}
Let $p; q \refines r$.\\
We are to prove $(p | s); q \refines r | s$.\\
$(p | s); q = (s | p); q \refines s | (p ; q) \refines s | r = r | s$
\end{proof}

\begin{theorem}
Right locality implies Milner frame rule.
\end{theorem}

\begin{proof}
Let $q; r \refines p$.\\
We are to prove $q; (r | s) \refines p | s$.\\
$q; (r |s) \refines (q; r) | s \refines p | s$.\\
\end{proof}

\begin{theorem}
Hoare frame rule implies left locality.
\end{theorem}

\begin{proof}
Let $p; q \refines r$ implies $(p | s); q \refines r | s$.\\
Let $r = (p; q)$ then the premise of the rule is true.\\
So $(p | s); q = (s | p); q \refines (p; q) | s = s | (p; q)$.
\end{proof}

\begin{theorem}
Milner frame rule implies right locality.
\end{theorem}

\begin{proof}
Let $q; r \refines p$ implies $q; (r | s) \refines p | s$.\\
Let $p = (q; r)$ then the premise of the rule is true.\\
So $q; (r | s) \refines (q; r) | s$.
\end{proof}

\subsection*{Separation logic}

\begin{axiom}[Commutativity and associativity of $|$]
\end{axiom}

Do we need it?

\begin{law}[Exchange]
$(p|p');(q|q') \refines (p;q)|(p';q')$
\end{law}

\begin{rul}[Parallel for Hoare]
$\{p\}q\{r\}$ and $\{p'\}q'\{r'\}$ imply $\{p|p'\} q|q' \{r|r'\}$
\end{rul}

\begin{rul}[Parallel for Milner]
$\milner{p}{q}{r}$ and $\milner{p'}{q'}{r'}$ imply $\milner{p | p'}{q | q'}{r | r'}$.
\end{rul}

\begin{theorem}
Exchange law implies parallel rules for Hoare and Milner.
\end{theorem}

\begin{proof}
Let $p; q \refines r$ and $p'; q' \refines r'$.\\
We are to prove $(p | p'); (q | q') \refines r | r'$.\\
$(p | p'); (q | q') \refines (p; q) | (p'; q') \refines r | r'$.\\
Let $q; r \refines p$ and $q'; r' \refines p'$.\\
We are to prove $(q | q'); (r | r') \refines p | p'$.\\
$(q | q'); (r | r') \refines (q; r) | (q'; r') \refines p | p'$.
\end{proof}

\begin{theorem}
Parallel rule for Hoare implies exchange law.
\end{theorem}

\begin{proof}
Let $p; q \refines r$ and $p'; q' \refines r'$ imply $(p | p'); (q | q') \refines r | r'$\\
Let $r = (p; q)$ and $r' = (p'; q')$ then the premise of the rules is true.\\
So $(p | p'); (q | q') \refines (p; q) | (p'; q')$.\\
\end{proof}

\begin{theorem}
Parallel rule for Milner implies exchange law.
\end{theorem}

\begin{proof}
Let and $q; r \refines p$ and $q'; r' \refines p'$ imply $(q | q'); (r | r') \refines p | p'$.\\
Let $p = (q; r)$ and $p' = (q'; r')$ then the premise of the rules is true.\\
So $(q | q'); (r | r') \refines (q; r) | (q'; r')$.\\
\end{proof}

\subsection*{Weakest preconditions}

\begin{definition}
$wp(q, r) = \textrm{largest} p$ such that $\{p\}q\{r\}$.
That is, we postulate that $\{wp(q, r)\} q \{r\}$ and 
for all $p'$ $\{p'\}q\{r\}$ implies $p' \refines wp(q, r)$.
\end{definition}

\begin{law}[Adjointness]
$p \refines wp(q, r)$ iff $p; q \refines r$
\end{law}

\begin{rul}[WP1]
$wp(q, r); q \refines r$
\end{rul}

\begin{rul}[WP2]
$p \refines wp(q, (p; q))$
\end{rul}

\begin{theorem}
Adjointness law implies WP1 and WP2.
\end{theorem}

\begin{proof}
By definition $wp(q, r); q \refines r$.\\
\\
Let $r = (p; q)$\\
$p; q \refines p; q$\\
then $p \refines wp(q, r) = wp(q, (p; q))$.
\end{proof}

\begin{theorem}
WP1 and WP2 imply adjointness law.
\end{theorem}

\begin{proof}
Let $p; q \refines r$\\
$p \refines wp(q, (p; q))$ means that\\
$wp(q, (p; q)); q \refines p; q \refines r$\\
$wp(q, r); q \refines r$ means that\\
$wp(q, r); q \refines r$
\end{proof}


%% wp properties 

\begin{theorem}[wp monotonicity]
$q' \refines q$ and $r \refines r'$ imply $wp(q, r) \refines wp(q', r')$
\end{theorem}

\begin{proof}
\end{proof}

\begin{theorem}[Stepwise wp]
$wp((q;q'), r) \refines wp(q, wp(q', r))$
\end{theorem}

\begin{proof}
$wp((q; q'), r); (q; q') \refines r$\\
$wp(q, wp(q', r)); q \refines wp(q'; r)$\\
$wp(q'; r); q' \refines r$
\end{proof}

\begin{theorem}
$wp(q, r) \refines wp((q;s), (r;s))$
\end{theorem}

\begin{proof}
\end{proof}

\end{document}