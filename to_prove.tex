\documentclass{article}
\usepackage{amssymb}
\usepackage{amsmath}

\begin{document}

\newcommand{\refines}{\sqsubseteq}
\newcommand{\eps}{\varepsilon}

\newcommand{\plotkin}[1]{\stackrel{#1}{\rightarrow}}
\newcommand{\milner}[1]{\stackrel{#1}{\leadsto}}


\newtheorem{axiom}{Axiom}
\newtheorem{law}{Law}
\newtheorem{rul}{Rule}  % because 'rule' is already occupied
\newtheorem{theorem}{Theorem}
\newtheorem{definition}{Definition}

Hello!

Laws and rules are mere shortcuts for some statements.
Only theorems are supposed to be proven (and they typically state that
some rule implies some law or vice versa).


\begin{axiom} [$\refines$ is partial order]
\end{axiom}

\begin{axiom} [Monotonicity of $;$]
\end{axiom}


\subsection*{Associativity and sequence rules}

\begin{law}[Assoc. of ;]
$p;(q;r) = (p;q);r$
\end{law}

\begin{rul}[Sequence rule for Hoare]
$\{p\}q\{s\}, \{s\}q'\{r\}$ imply $\{p\}q;q'\{r\}$
\end{rul}

Same rules for Milner, Plotkin, test.

\begin{theorem}
Associativity law implies all sequence rules.
\end{theorem}

\begin{theorem}
Sequence rules for Hoare+Plotkin or Milner+test imply associativity law.
\end{theorem}


\subsection*{Skip}

\begin{law}[Skip]
$p;\eps = p = \eps;p$
\end{law}

\begin{rul}[Nullity for Hoare]
$\{p\}\eps\{p\}$
\end{rul}

Same for Plotkin, Milner, test.

\begin{theorem}
Skip law implies all nullity rules.
\end{theorem}

\begin{theorem}
Conjunction of all nullity rules implies skip law.
\end{theorem}


\subsection*{Distribution and nondeterminism}

\begin{definition}[$\sqcup$]
\end{definition}

\begin{law}[Left distributivity]
$p; (q \sqcup q') = p;q \sqcup p;q'$
\end{law}

\begin{rul}[Nondeterminism for Hoare]
$\{p\}q\{r\}, \{p\}q'\{r\}$ imply $\{p\}q\sqcup q'\{r\}$
\end{rul}

\begin{rul}[Nondeterminism for Milner]
$p \milner{q} r, p \milner{q'} r$ imply $p \milner{q\sqcup q'} r$
\end{rul}

\begin{rul}[Nondeterminism for Plotkin]
$p \plotkin{q} r$ or $p \plotkin{q'} r$ implies $p \plotkin{q\sqcup q'} r$
\end{rul}

\begin{rul}[Nondeterminism for test]
$p [q] r$ or $p [q'] r$ implies $p [q;q'] r$
\end{rul}

\begin{theorem}
Left distributivity implies nondeterminism for all four formalisms.
\end{theorem}


\begin{theorem}
Nondeterminism rules for (Hoare or Milner) and (Plotkin or test) 
imply left distributivity.
\end{theorem}


\subsection*{Locality and frame rules}
Theorems seem trivial.

\subsection*{Separation logic}

\begin{axiom}[commutativity and associativity of $|$]
\end{axiom}

Do we need it?

\begin{law}[Exchange]
$(p|p');(q|q') \refines (p;q)|(p';q')$
\end{law}

\begin{rul}[Parallel for Hoare]
$\{p\}q\{r\}$ and $\{p'\}q'\{r'\}$ imply $\{p|p'\} q|q' \{r|r'\}$
\end{rul}

\begin{rul}[Parallel for Milner]
Likewise.
\end{rul}

\begin{theorem}
Exchange law implies parallel rules for Hoare and Milner.
\end{theorem}

\begin{theorem}
Parallel rules for Hoare and Milner imply exchange law.
\end{theorem}


\subsection*{Weakest preconditions}

\begin{definition}
$wp(q, r) = \textrm{largest } p$ such that $\{p\}q\{r\}$.
\end{definition}

\begin{law}[adjointness]
$p\refines wp(q, r)$ iff $p;q \refines r$
\end{law}

Do we need  definition (because it seems that law is enough)?

\begin{rul}[WP1]
$wp(q, r); q \refines r$
\end{rul}

\begin{rul}[WP2]
$p \refines wp(q, (p;q))$
\end{rul}

\begin{theorem}
Adjointness law implies WP1 and WP2.
\end{theorem}

\begin{theorem}
WP1 and WP2 imply adjointness law.
\end{theorem}


%% wp properties 

\begin{theorem}[wp monotonicity]
$q_1 \refines q, r \refines r'$ imply $wp(q, r) \refines wp(q_1, r')$
\end{theorem}

\begin{theorem}[stepwise wp]
$wp((q;q'), r) \refines wp(q, wp(q', r))$
\end{theorem}

\begin{theorem}
$wp(q, r) \refines wp((q;s), (r;s))$
\end{theorem}

\end{document}